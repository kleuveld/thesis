% %%%%%%%%%%%%%%%%%%%%%%%%%%
%\pagestyle{fancy}
\addcontentsline{toc}{chapter}{Acknowlegdements}
\chapter*{Acknowlegdements}
%%%%%%%%%%%%%%%%%%%%%%%%%%
First of all, I would like to thanks Dieudonnee Mundi; not only were his knowledge and helpfulness essential to the data collection of me and other researchers around Bukavu, Mundi's gracious hospitality made us all feel like part of his family, which made long stays in Bukavu something to look forward to. His untimely death leaves a gap for the research community that will be extremely difficult to fill.

My family has been important throughout. This includes my parents and brothers, but especially my partner, Aster, who has been supportive of me finally finishing this project. Special thanks are in order for Vanja: even though she is too young to have seen much of the work that has gone in this thesis, she was an enormous inspiration in finishing it.

During field work, I have had great research assistance from Eva, Evelien, Biho, Imke, Karin, Romy, Alexandra, John and Megan. The field work in Congo was done in close cooperation with the Université Catholique Bukavu, in particular Janvier, and the Université du Graben, where Kennedy has been a great help. In Cameroon, we worked closely with the Institut National de la Statistique, who did tremendous work in making field work possible. I have had even more assistance from many interviewers in all of these countries, especially Chance Jacques, Papy, Eustache, Freddy, Raymond, Bakari and Aisha. 

Science is a team sport, and fortunately the DEC department is filled with great team players. Of great support to me have been my supervisors Maarten and Marrit, and my promotor Erwin. 

My co-authors have also been a joy to work with. I've spent months in Cameroon with Niccolò, who is surprisingly knowledgeable about the Cameroonian musical genres of Makossa and Bikutsi. In Sierra Leone, I spent much time with Maarten and Francesco, both of whom have contributed to more valuable insights than could have possibly been included in a thesis, including thorough (though perhaps not entirely emprically grounded) investigations into the arboreal feeding habits of Choeropsis Liberiensis and into the unexpected relationship between various local species of Cephalophinae, urban crime and inter-city transport within Sierra Leone. With my DEC co-authors on Chapter 3, Martha, Janneke and Lonneke, I've spent only a short time in Bukavu, but fortunately more time in the Netherlands. Martha especially has been a frequent guest at our house and has been  a great inspiration: not only on getting that PhD finished, but also on getting a dog (said dog has helped greatly in keeping sane while getting that PhD finished).

There are many folks at DEC who are not co-authors, but have contributed greatly to the atmosphere that makes it such a good place to work. This includes Zihan, Paul, Aussi, Jan, Maria, Landry, Haki, Ian, Gonne, and all others I've spent so much time with during coffee break, group drinks and football games. One more reason DEC is such a great place to work, is the fabulous support staff, who make it possible to navigate the bureacratic maze that the university can be. So big thanks are in order for Marian, Dineke, Gré, and Betty.

Finally, I have combined finalizing this PhD project with jobs at EDI and the VU. While combining a PhD with a job was difficult at times, my colleagues at both these places have made it a joy.