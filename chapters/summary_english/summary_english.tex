%%%%%%%%%%%%%%%%%%%%%%%%%%
\clearpage
\addcontentsline{toc}{chapter}{Summary}
\chapter*{Summary}
% %%%%%%%%%%%%%%%%%%%%%%%%%%

The global poor are increasingly concentrated in a limited number of countries. The World Bank expects that by 2030, up to two thirds of the world's extremely poor live in Fragile Conflict-affected settings, and mostly in rural areas. This thesis aims to investigate the local dynamics that underlie (or are caused by) the lack of development. The core argument in this thesis is that the effect of each of these risks and opportunities on development outcomes is not direct. 

%chapter 2
Chapter \ref{chap:slfootball} is based on fieldwork around a street football tournament in Kenema, Eastern Sierra Leone. Using a set of lab-in-the-field experiments, we evaluate the impact of exposure to violent conflict on competitive behaviour. We find that conflicted affected youth in our sample are more likely to get a yellow or red card during the football tournament, are less risk averse, display more pro-social behaviour to their teammates and are more competitive towards their opponents. These findings complement a growing literature on the relationship between conflict and (pro-social) behaviour. This effect of conflict on behaviour may have consequences for long-term development.

%chapter 3
Chapter \ref{chap:n2a_impact} examines the impact of a program that provides subsidized inputs to smallholder farmers in Eastern DRC using a field experiment. We find that two agricultural seasons after the subsidy program, the use of inputs remained higher in the communities receiving the subsidy compared to those that did not. Fertilizer was increased by five percentage points, while the use of inoculant (a novel nitrogen-fixing technology) was increased by three percentage points. Given the low initial input use in the sample, these increases are substantial, and the fact they persist two seasons after the provision of the subsidies points to a structural improvement of adoption. However, we do not find evidence of increased yields or improved food security. Furthermore, input use further away from market towns was not affected, suggesting that the success of such programs highly depend on the context.

%chapter 4
Chapter \ref{chap:cameroontrust} explores the relationship of sending behaviour in an Investment Game and exposure to markets, a common indicator of trust. We use the results of an Investment Game played with over 3,000 rural household heads in Northern Cameroon. We find that, on average, respondents in Cameroon are less driven by expectations of reciprocity (trust) and more by social preferences than respondents in previous studies, often done using populations of university students. However, when we split our sample in a group with market experience and one without, we find that expectations do drive sending behaviour in the market group. There may be a learning effect where increased interactions on markets (often with strangers) may lead people to be more willing to act on their expectations.

%chapter 5
Chapter \ref{chap:congogbv} explores the drivers of SGBV in Easter Congo by examining the characteristics of the victims. To measure victimization without suffering from social desirability bias I conducted a list experiment. I find high rates of victimization: 30\% of the women in my sample have been victimized in the 12 months prior to the interview. The victims are likely to be married to higher-status men, have low intra-household bargaining power, and have been exposed to violent conflict to the extent where they have lost family or household members before 2012 (two years before the list experiment). These results are not in line with the view that SGBV is mostly caused by direct perpetration by armed groups. Rather, it suggests that there may be a long-term effect of violent conflict on intimate partner violence. This suggests that the problem of SGV in DRC can only be addressed by imporiving the position of women in Congolese households and society in general.


%chater 6: 
Chapter \ref{chap:conclusion} provides a synthesis of these results, linking them back to the original problem statement of poverty being increasingly concentrated in a limited number of countries. The chapter argues that this requires extensive local level data collection to ensure that development programs have the desired effect.