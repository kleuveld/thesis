\chapter{Conclusion}
\label{chap:conclusion}
%link back to general themes
The preceding chapters each presented local evidence on development. Each chapter discussed a subset of the of the risks and opportunities that I presented in Chapter \ref{chap:introduction}, and their impact on development, focusing on different countries: Sierra Leone, DRC and Cameroon. Data collection in such settings is difficult and costly, but detailed is needed to get detailed insights at the local level. Such a local perspective allowed me to asses how the effects of these risks and opportunities are mediated through social capital and behaviour, which is important as such mediation may have surprising, or unintuitive consequences for the impacts that these risks and opportunities have on development. In this concluding chapter, I will briefly summarize the findings of each chapter, and then provide a synthesis focusing on the implications of these findings for development policy.
%what does this section look like?

%\section{Conflict and Competitive Behaviour}
At the core of Chapter \ref{chap:slfootball} is the question whether conflict affects competitive behaviour among youths in Eastern Sierra Leone. While conflict is an import risk to development, competitive behaviour may play a role in fostering development. We invited football players from a youth tournament to participate in a post-game research activity comprising a short questionnaire and a number of behavioural experiments. We found that the larger the exposure to conflict a player had seen, the more competitively they behaved in an effort game when matched with players from opposing teams, and the more foul cards they received in the football tournament. This highlights how unexpected the behavioural links between conflict and development can be and points at a potential (minor) upside of conflict to development: it may transform society to become dynamic, and more competitive, thus promoting economic development.

%\section{Subsidies and Technology Adoption}
%What is the effect of input subsidies on novel technology adoption? (Chapter \ref{chap:n2a_impact}
An important component of development is agricultural productivity. However, in Africa this productivity is constrained by low adoption of inputs. Chapter \ref{chap:n2a_impact} revolves around the effectiveness of input subsidization as a way to remove this constraint. The chapter presents evidence from an impact evaluation of a project that provided subsidized input packages (including fertilizer, inoculant and seeds) to smallholder farmers in South Kivu province in the eastern DRC. The results suggest that the project effected an increase in use of both fertilizer and inoculant the year after the subsidized packages were provided. However, we were unable to find any impacts on yields and food security, perhaps because these are harder to measure with any precision. While these results suggest potential for subsidy programs, an important caveat is in place. When comparing the impact in villages that are close to markets to the impact in villages that are not, we find that the positive effects on input adoption is concentrated in villages closest to markets. This suggests that the project's success does not exist in a vacuum, but is context dependent. In this specific case, market access is an important enabling factor for the intervention.

%\section{Markets and Trust}
%What is the effect of market access on trusting behaviour (Chapter \ref{chap:n2a_impact}
While Chapter \ref{chap:n2a_impact} suggests an important role for markets in enabling supply of inputs, markets may have more indirect effects on development as well. In  Chapter \ref{chap:cameroontrust} we explore the effects of market exposure on behaviour in an investment game. Specifically, we consider three determinants of sending behaviour in the game (social preferences, expectations and risk preferences) and examine how the effects of these determinants vary between villages with and without market exposure. When considering these determinants across the entire research population, we find that our respondents are strongly motivated by social preferences, but we find no evidence that they are motivated by expectations. In other words, they consider the game an opportunity to transfer money to a fellow villager, rather than an investment opportunity to get money back. However, when separating villages with and without markets, we do find an effect of expectations in the market villages: respondents with market exposure are more likely to see the money sent as an investment. This suggests that the impacts of markets on development goes deeper than providing a mechanism for the efficient allocation of resources. The repeated interactions with (and perhaps dependence on) relative strangers changes the way people behave. This could be either through a learning effect, where people learn from the interactions with strangers and are thus more comfortable in trusting them, or that the framing of market interaction lead people to behave more rationally \citep[see e.g.][]{List2008,Cecchi2013}, and thus more likely to send money to people if they expect them to return some of it .

%\section{Drivers of SGBV in DRC}
%What are the drivers of sexual and gender-based violence in Eastern Congo (Chapter \ref{chap:congogbv})
In chapter \ref{chap:congogbv}, I analysed the results from a list experiment, in order to identify potential drivers {}of SGBV in Eastern Congo.I found high rates of incidence of SGBV: I estimate that 30\% of the women in the sample were the victim of SGBV in the twelve months preceding data collection. I find that incidence rates are higher among women married to higher-status men, among women who have low intra-household bargaining power, and among women with a history of conflict. I find no relation between recent conflict and SGBV victimization. These findings go against the common narrative where SGBV in DRC is framed as ``weapon of war''. While I do find evidence of a link between violent conflict and SGBV, addressing the high incidence rates takes more than an end to violent conflict. A first condition should be to improve the position of women in Congolese society. The fact that women who have attended secondary school are hardly ever victimized by SGBV may point at an effective strategy, but more research is needed to prove a causal link.

%\section{Concluding Remarks}
%Disccussion: are the methods relevant?
A key argument in this thesis has been the importance of conducting empirical research at the local level. So how do these findings differ from the common wisdom based on national-level experiences? Does our appreciation of the impact that markets, conflict and aid have on development change when assessing it through a local lens? As for conflict, the impact of conflict remains decidedly negative. But at the local level, the findings presented in Chapter \ref{chap:slfootball} complement a growing literature on the behavioural effects of conflict, not all of which are bad. Taken together with earlier findings that conflict increases in-group pro-social behaviour \cite{Bellows2009b,Voors2012,Gilligan2014,Bauer2014} and political engagement \cite{Bellows2009b,Blattman2009a}, our findings suggest that the experience of conflict has some positive effects on development which may be important to post-conflict recovery.

Despite these possible positive impacts on development, conflict remains associated with overwhelmingly negative outcomes. Chapter \ref{chap:congogbv} deals with one such outcome: SGBV. The rampant levels of SGVB place a huge burden on the Congolese people (it is important to note that even though the focus in this thesis is on women, victimization rates among men are high as well \citep{Christian2011}). While it is common to attribute this violence to the conflict, such a narrow framing risks missing the larger problems that persist within Congolese society \citep[see e.g.][]{Autesserre2012a}. The findings presented in Chapter \ref{chap:congogbv} suggest that violence persists long after the conflict has ended, and that in order to prevent violence, the position of women in Congolese society must be improved.

As for markets, their most obvious role in economic development is to improve economic efficiency by decreasing transaction costs. This is evidenced by the finding in Chapter \ref{chap:n2a_impact} that aid has more impact closer to markets: after all, farmers with good market access can more easily obtain inputs and sell outputs. However, this is not the only impact: markets also increase the number of interactions with strangers, which requires a different set of norms, and a different set of expectations, than dealing with your close kin and neighbours. The findings from Chapter \ref{chap:cameroontrust} are one example of the benefits this could have: increased levels of trust between members of the community. This is in line with existing literature on the effect that large-scale societies have on social preferences \citep{Henrich2010}.

Finally, development aid. Unlike conflict and markets, development aid provides a policy lever with scope for adjustment. And the findings presented above have various implications for how to ensure we properly use this policy lever. The evidence presented in Chapter \ref{chap:n2a_impact} suggests that relatively light interventions can produce results, even in difficult areas such as DRC. The findings also suggest that the project did not achieve impact everywhere, suggesting that better targeting is in order. Such targeting can only be effective given good and reliable data. Another example of how data can better guide development policy flows from the findings of Chapter \ref{chap:congogbv}: where a large part of the development community may view conflict as the sole driver of SGBV, detailed micro-level data suggests reality may be more complex, and require different policies to address the problem. Of course not every detailed micro-level finding has great implications for development policy. The unexpected benefits of conflict and markets as presented in chapters \ref{chap:slfootball} and \ref{chap:cameroontrust} respectively may be interesting for academics, but they may be too small for a development organization to meaningfully engage with. However, like the respondents from Chapter \ref{chap:cameroontrust} behaved differently than the students commonly used in lab settings, people in a different conflict, or at a different point in a market exposure gradient, may change their behaviour more drastically. Development organizations fully ignoring the possibility of such unforeseen benefits thus may be leaving money on the table.

This thesis started out with the observation that the extreme poverty in the world is increasingly concentrated in a small number of countries, defying global trends of increasing prosperity. The arguments outlined above mean we cannot simply make assumptions about the relationships between the risks and opportunities that these countries face, and their development trajectory. Conflict does not only have downsides, SGBV in conflict areas will not be solved through peace treaties and markets do more than just allocate goods. This means that development experts (practitioners and academics alike) should be modest about their knowledge of what drives poverty. We should be diligent about checking the assumptions behind development policy, and monitor their effect, so that efforts can be focused where the needs are greatest, and the potential for impact largest. Such data collection is costly and often risky, but the alternative may be basing our policy based on faulty assumptions (such as the ``weapon war'' view of SGBV discussed in this thesis), leading to less effective aid. In this way, such faulty assumptions are more costly than data collection, and those costs will be borne by those who depend on aid, who are often the most vulnerable. 

\bibliographystyle{chicago}
%path to .bib file (e.g. automatically exported by mendeley) DO NOT include the file extension!
\bibliography{C:/Users/kld330/SurfDrive2/Data/BibTeX/Thesis}
%\bibliography{\bibtex/Thesis}