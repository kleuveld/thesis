\chapter{Conclusion}
\label{chap:conclusion}
%link back to general themes
This dissertation aims to investigate the link between local risks and opportunity, such as conflict, development aid and market access, and development, and in particular the way this link is mediated by social capital and behaviour. The four main chapters examined local evidence from three different countries: Sierra Leone, DRC and Cameroon. These links are complex, and each chapter



%what does this section look like?



\section{Conflict and Competitive Behaviour}
%What is the relationship between conflict and competitive behaviour? (Chapter \ref{chap:slfootball});
%The main contribution of this study is to provide insight into the determinants of competitive behavior and its relation with exposure to violent conflict.

At the core of Chapter \ref{chap:slfootball} is the question whether conflict affects competitive behaviour among youths in Eastern Sierra Leone. Conflict is an import risk to development, and competitive behaviour may play a role in development. We invited football players from a youth tournament to participate in a post-game research activity comprising a short questionnaire and a number of behavioural experiments. We found that the larger the exposure to conflict a player had seen, the more competitively they behaved in an effort game when matched with players from opposing teams, and the more foul cards they received in the football tournament. This highlights how unexpected the behavioural links between conflict and development can be and points at a potential upside of conflict to development: it may transform society to become dynamic, and more competitive, thus promoting economic development.

\section{Subsidies and Technology Adoption}
%What is the effect of input subsidies on novel technology adoption? (Chapter \ref{chap:n2a_impact}
Low adoption of  inputs is a major constraint to agricultural productivity in Africa. Chapter \ref{chap:n2a_impact} revolves around the effectiveness of input subsidization as a way to remove this constraint. The chapter presents evidence from an impact evaluation of a project that provided subsidized input packages (including fertilizer, inoculant and seeds) to smallholder farmers in South Kivu province in the eastern DRC. The results suggest that the project effected an increase in use of both fertilizer and inoculant the year after the subsidized packages were provided. However, we were unable to find any impacts on yields and food security, perhaps because these are harder to measure with any precision. While these results suggest potential for subsidy programs, an important caveat is in place. When comparing the impact in villages that are close to markets to the impact in villages that are not, we find that the positive effects on input adoption is concentrated in villages closest to markets, suggesting that market access is an important enabling factor. \todo{beschrijf de heterogeneous effects wat alemgener?} 

\section{Markets and Trust}
%What is the effect of market access on trusting behaviour (Chapter \ref{chap:n2a_impact}
While Chapter \ref{chap:n2a_impact} suggests an important role for markets in enabling supply of inputs, markets may have more indirect effects on development as well. In  Chapter \ref{chap:cameroontrust} we explore the effects of market exposure on behaviour in an investment game. Specifically, we consider three determinants of sending behaviour in the game (social preferences, expectations and risk preferences) and examine how the effects of these determinants vary between villages with and without market exposure. We find that when considering these determinants across the entire research popultation, we find that our respondents are strongly motivated by social preferences, but we find no evidence that they are motivated by expectations. In other words, they consider the game an opportunity to transfer money to a fellow villager, rather than an investment opportunity to get money back. However, when separating villages with and without markets, we do find an effect of expectations in the market villages: respondents with market exposure are more likely to see the money sent as an investment. This suggests that the impacts of markets on development goes deeper than providing a mechanism for the efficient allocation of resources. The repeated interactions with (and perhaps dependence on) relative strangers changes the way people behave. \todo{expand: how is behaviour changed. Citaatje zoeken in paper om te onderbouwen}


\section{Drivers of SGBV in DRC}
%What are the drivers of sexual and gender-based violence in Eastern Congo (Chapter \ref{chap:congogbv})
In chapter \ref{chap:congogbv}, I analysed the results from a list experiment, in order to identify potential drivers of SGBV in Eastern Congo. I found high rates of incidence of SGBV: I estimate that 30\% of the women in the sample were the victim of SGBV in the twelve months prior to data collection. I find that incidence rates are higher among women married to higher-status men, among women who have low intra-household bargaining power, and among women with a history of conflict. I find no relation between recent conflict and SGBV victimization. These findings go against the common narrative where SGBV in DRC is framed as ``weapon of war''. While I do find evidence of a link between violent conflict and SGBV, addressing the high incidence rates takes more than an end to violent conflict. A first condition should be to improve the position of women in Congolese society. The fact that women who have attended secondary school are hardly ever victimized by SGBV may point at an effective strategy, but more research is needed to prove a causal link.

\section{Concluding Remarks}
%Disccussion: are the methods relevant?
A key argument in this dissertation has been the importance of conducting empirical research at the local level. So how do these finding differ from the common wisdom based on national-level experiences? Does our appreciation of the impact that markets, conflict and aid have on development change when assessing it through a local lens? As for conflict, the impact of conflict remains decidedly negative. But at the local level, the findings presented in Chapter \ref{chap:slfootball} complement a growing literature on the behavioural effects of conflict, not all of which are bad. Taken together with earlier findings that conflict increases in-group pro-social behaviour \cite{Bellows2009b,Voors2012,Gilligan2014,Bauer2014} and political engagement \cite{Bellows2009b,Blattman2009a}, our findings suggest that the experience of conflict has some positive effects on development which may be important to post-conflict recovery.

Despite these possbile positive impacts on development, conflict remains assocatiated with overwhelmingly negative outcomes. Chapter \ref{chap:congogbv} deals with one such outcome: SGBV. The rampant levels of SGVB place a huge burden on the Congolese people (it is important to note that even though the focus in this thesis is on women, victimization rates among men are high as well \todo{cite}). While it is common to attribute this violence to the conflict, such a narrow framing risks missing the larger problems that persist within Congolese society \citep{Autesserre2012a}. The findings presented in Chapter \ref{chap:congogbv} suggest that violence persists long after the conflict has ended, and that improving the position of women in Congolese society can play an important role in preventing violence.

As for markets, their main role in economic development is to improve economic efficiency by decreasing transcation costs \todo{cite me papi}. This has far-reaching implications, as is evidenced by the finding in Chapter \ref{chap:n2a_impact} that aid has more impact closer to markets: after all, farmers with good market access can more easily obtain inputs and sell outputs. There is however a different, more subtle effect. Markets increase the number of interactions with strangers, which requires a different set of norms, and a different set of expectations, than dealing with your family and close neighbors. \todo{expand}

What does mean in terms of policy? There is not much scope to change policies regarding conflict and market access: I won't argue for more conflict, and the importance of markets is well-established. Development aid, however, is an area where there are policy options. 

Markets: distance or investment?

%Policy relevance.
These results matter. While knowing the cost of conflict will not alter the appreciation that those in development circles have for conflict -- conflict will always be a net negative, after all -- a better understanding of the effects will allow us to more effectively implement interventions aimed at post-conflict recovery. 



\bibliographystyle{chicago}
%path to .bib file (e.g. automatically exported by mendeley) DO NOT include the file extension!
\bibliography{C:/Users/kld330/SurfDrive2/Data/BibTeX/Thesis}
%\bibliography{\bibtex/Thesis}