\chapter{Conclusion}
\label{chap:conclusion}

\section{Introduction}
%link back to general themes
This dissertation aimed to investigate the link between local risks and opportunity, such as conflict, development aid and market access, and development, and in particular the way this link is mediated by social capital and behaviour. The four main chapters examined local evidence from three different countries: Sierra Leone, DRC and Cameroon. These links are complex, and each chapter



%what does this section look like?



\section{Conflict and Competitive Behaviour}
%What is the relationship between conflict and competitive behaviour? (Chapter \ref{chap:slfootball});
%The main contribution of this study is to provide insight into the determinants of competitive behavior and its relation with exposure to violent conflict.

At the core of Chapter \ref{chap:slfootball} is the question whether conflict affects competitive behaviour among youths in Eastern Sierra Leone. Conflict is an import risk to development, and competitive behaviour may play a role in development. We invited football players from a youth tournament to participate in a post-game research activity comprising a short questionnaire and a number of behavioural experiments. The chapter found that the larger the exposure to conflict a player had seen, the more competitively they behaved in an effort game when matched with players from opposing teams, and the more foul cards they received in the football tournament. This highlights how unexpected the behavioural links between conflict and development can be and points at a potential upside of conflict to development: it may transform society to become dynamic, and more competitive, thus promoting economic development.

\section{Subsidies and Technology Adoption}
%What is the effect of input subsidies on novel technology adoption? (Chapter \ref{chap:n2a_impact}
Low adoption of  inputs is a major constraint to agricultural productivity in Africa. Chapter \ref{chap:n2a_impact} revolves around the effectiveness of input subsidization as a way to remove this constraint. The chapter presents evidence from an impact evaluation of a project that provided subsidized input packages (including fertilizer, inoculant and seeds) to smallholder farmers in South Kivu province in the eastern DRC. The results suggest that the project succeeded in effecting an increase in use of both fertilizer and inoculant the year after the subsidized packages were provided. However, my co-authors and I were unable to find any impacts on yields and food security, perhaps because these are harder to measure with any precision. The results suggest that development aid 

\section{Markets and Trust}
%What is the effect of market access on trusting behaviour (Chapter \ref{chap:n2a_impact}
While markets can provide inputs, such as fertlizer, 

\section{Drivers of SGBV in DRC}
%What are the drivers of sexual and gender-based violence in Eastern Congo (Chapter \ref{chap:congogbv})

\section{Concluding Remarks}
%Disccussion: are the methods relevant?


%Policy relevance.
These results matter. While knowing the cost of conflict will not alter the appreciation that those in development circles have for conflict -- conflict will always be a net negative, after all -- a better understanding of the effects will allow us to more effectively implement interventions aimed at post-conflict recovery. 


