\chapter{Conclusion}
\label{chap:conclusion}

\section{Introduction}
%link back to general themes
This dissertation aimed to investigate the link between local risks and opportunity, such as conflict, development aid and market access, and development, and in particular the way this link is mediated by social capital and behaviour. The four main chapters examined local evidence from three different countries: Sierra Leone, DRC and Cameroon. These links are complex, and each chapter



%what does this section look like?



\section{Conflict and Competitive Behaviour}
%What is the relationship between conflict and competitive behaviour? (Chapter \ref{chap:slfootball});
%The main contribution of this study is to provide insight into the determinants of competitive behavior and its relation with exposure to violent conflict.

At the core of Chapter \ref{chap:slfootball} is the question whether conflict affects competitive behaviour among youths in Eastern Sierra Leone. Conflict is an import risk to development, and competitive behaviour may play a role in development. We invited football players from a youth tournament to participate in a post-game research activity comprising a short questionnaire and a number of behavioural experiments. The chapter found that the larger the exposure to conflict a player had seen, the more competitively they behaved in an effort game when matched with players from opposing teams, and the more foul cards they received in the football tournament. This highlights how unexpected the behavioural links between conflict and development can be and points at a potential upside of conflict to development: it may transform society to become dynamic, and more competitive, thus promoting economic development.

\section{Subsidies and Technology Adoption}
%What is the effect of input subsidies on novel technology adoption? (Chapter \ref{chap:n2a_impact}
Low adoption of  inputs is a major constraint to agricultural productivity in Africa. Chapter \ref{chap:n2a_impact} revolves around the effectiveness of input subsidization as a way to remove this constraint. The chapter presents evidence from an impact evaluation of a project that provided subsidized input packages (including fertilizer, inoculant and seeds) to smallholder farmers in South Kivu province in the eastern DRC. The results suggest that the project succeeded in effecting an increase in use of both fertilizer and inoculant the year after the subsidized packages were provided. However, my co-authors and I were unable to find any impacts on yields and food security, perhaps because these are harder to measure with any precision. While these results suggest potential for subsidy programs, an important caveat is in place. When comparing the impact in villages that are close to markets to the impact in villages that are not, we find that the positive effects on input adoption is concentrated in villages closest to markets, suggesting that market access is an important enabling factor. 

\section{Markets and Trust}
%What is the effect of market access on trusting behaviour (Chapter \ref{chap:n2a_impact}
While Chapter \ref{chap:n2a_impact} suggests an important role for markets in enabling supply of inputs, markets may have more indirect effects on development as well. In  Chapter \ref{chap:cameroontrust} we explore the effects of market exposure on behaviour in an invest game. Specifically, we consider three determinants of sending behaviour (social preferences, expectations and risk preferences) and examine how the effects of these determinants varies between villages with and without market exposure. We find that in general, our respondents are strongly motivated by social preferences, but we find no evidence that they are motivated by expectation. In other words, they consider the game an opportunity to transfer money to a fellow villager, rather than an investment opportunity to get money back. However, when separating villages with and without markets, we do find an effect of expectations in the market villages: respondents with market exposure are more likely to see the money sent as an investment.


\section{Drivers of SGBV in DRC}
%What are the drivers of sexual and gender-based violence in Eastern Congo (Chapter \ref{chap:congogbv})
In chapter \ref{chap:congogbv}, I analysed the results from a list experiment, in order to identify potential drivers of SGBV in Eastern Congo. I found high rates of incidence of SGBV: I estimate that 30\% of the women in the sample were the victim of SGBV in the twelve months prior to data collection. I find that incidence rates are higher among women married to higher-status men, among women who have low intra-household bargaining power, and among women with a history of conflict. I find no relation between recent conflict and SGBV victimization. These findings go against the common narrative where SGBV in DRC is framed as ``weapon of war''. While I do find evidence of a link between violent conflict and SGBV, addressing the high incidence rates takes more than an end to violent conflict. A first condition should be to improve the position of women in Congolese society. The fact that women who have attended secondary school are hardly ever victimized by SGBV may point at an effective strategy, but more research is needed to prove a causal link.



\section{Concluding Remarks}
%Disccussion: are the methods relevant?
In these four examples, advice that holds true at a high level, is challenged when looking in finer detail. Chapters \ref{chap:slfootball} and \ref{chap:congogbv} both deal with local impacts of conflict. While conflict is inherently destructive 

While conflict is terrible, at a local level it may have positive effects that good development policy should focus on. At the same time, some of the negatives associated with violent conflict (like SGBV) may persist and need structural interventions to address. 

%
Markets: distance or investment?



%Policy relevance.
These results matter. While knowing the cost of conflict will not alter the appreciation that those in development circles have for conflict -- conflict will always be a net negative, after all -- a better understanding of the effects will allow us to more effectively implement interventions aimed at post-conflict recovery. 



\bibliographystyle{chicago}
%path to .bib file (e.g. automatically exported by mendeley) DO NOT include the file extension!
\bibliography{C:/Users/kld330/SurfDrive2/Data/BibTeX/Thesis}
%\bibliography{\bibtex/Thesis}