\chapter{Conclusion}
\label{chap:conclusion}

\section{Introduction}
%link back to general themes
The relationship between risks (such as conflict) and opportunities (such as markets and aid) on the one hand, and development on the other is a complex one. In this dissertation I have examined how such relationships are mediated by social capital and institutions in three settings: Sierra Leone, Eastern DRC and Northern Cameroon.


%what does this section look like?



\section{Conflict and Competitive Behaviour}
%What is the relationship between conflict and competitive behaviour? (Chapter \ref{chap:slfootball});

At the core of Chapter \ref{chap:slfootball} was the question whether conflict affected competitive behaviour among youths in Eastern Sierra Leone. Conflict is an import risk to development, and competitive behaviour may play a role in development. We invited football players from a youth tournament to participate in a post-game research activity comprising a short questionnaire and a number of behavioural experiments. The larger the exposure to conflict a player had seen, the more competitively they behaved in an effort game. This higlights how unexpected the behavioural links between conflict and development can be.


%Conflict leads to competitive behaviour. The end.

\section{Subsidies and Technology Adoption}
%What is the effect of input subsidies on novel technology adoption? (Chapter \ref{chap:n2a_impact}

\section{Markets and Trust}
%What is the effect of market access on trusting behaviour (Chapter \ref{chap:n2a_impact}

\section{Drivers of SGBV in DRC}
%What are the drivers of sexual and gender-based violence in Eastern Congo (Chapter \ref{chap:congogbv})

\section{Concluding Remarks}
%Disccussion: are the methods relevant?


%Policy relevance.
These results matter. While knowing the cost of conflict will not alter the appreciation that those in development circles have for conflict -- conflict will always be a net negative, after all -- a better understanding of the effects will allow us to more effectively implement interventions aimed at post-conflict recovery. 


